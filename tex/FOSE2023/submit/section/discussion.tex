\section{考察\label{discussion}}
  \begin{figure}[t]
    \begin{verbatim}
      =======
      Suggestion 1

      def alloy(a,b):
          if a == 0:
              return 'Silver'
          elif b == 0:
              return 'Gold'
          else:
              return 'Alloy'

      =======
      Suggestion 2

      def problem212_a():
          a, b = map(int, input().split())
          if a > 0 and b == 0:
              print("Gold")
          elif a == 0 and b >

      =======
      Suggestion 3

      def main():
          a,b = map(int, input().split())
          if a > 0 and b == 0:
              print("Gold")
          elif a == 0 and b > 0:
    \end{verbatim}
    \caption{問題212-Aの中国版に対する推薦コードの一例}
    \label{recommend_212_A_en}
  \end{figure}
  \ref{result}章で示した結果について,現時点での筆者の考察を述べる.
  これらは筆者の考察であるため,今後の実験により検証されなければならない.
  \vspace{-1zh}
  \subsection{推薦コードの生成数と正答率の関係}
  表\ref{En_accuracy}, \ref{Ja_accuracy}, \ref{Zh_accuracy}のうち,
  各言語の$Accuracy$の差が3.0\%以上だった問題の最大値に着目すると,推薦された全コード数が5回のうち最も少なく,
  このことから,推薦1回につき推薦コードを最大10個取得することで,かえって$Accuracy$が低下した可能性が考えられる.

  また,難易度別に推薦された全コード数に着目すると,A問題はB, C, D問題に比べて,推薦された全コード数が最も少なかった.
  これはA問題が基本的な文法を問う問題であるため,処理が単純であり,
  類似した文面のコードが推薦されると重複として除外されている可能性があると考えた.

  さらに,言語別で推薦された全コード数の平均と,日本語のデータセットの$Accuracy$が最も高かったことより,
  日本語での入力により,Copilotがより最適なコードのみを推薦していることがわかる.
  この理由として,AtCoder\cite{AtCoder}が日本で運営されているため,
  多くの日本人がAtCoderを使用しており,日本語のコメントを含んだ回答がよりアップロードされ,それらが学習時に多く使用されたためであると考えられる.
  \subsection{中国語データセットにおける正答率の低下}
  表\ref{Med_accuracy}より,中国語のデータセットは,英語のデータセットに比べて,日本語のデータセットとの$Accuracy$の差が大きかった.
  これは,AtCoder\cite{AtCoder}において,日本語と英語の問題が準備されているため,
  日本語や英語のコメントを含んだより多くの正解コードが学習時に使用されたためであると考えられる.
  さらに中国語の生成コードが最も多かった原因としては,
  AtCoderの中国語版が存在しないため,中国語のコメントを含んだ正解コードの数が少なかった可能性がある.
  これに対して中国語を基とするプログラミングコンテストのデータセットを使用して調査を行う必要がある.
  
  また,A問題に関しては特に大きな差が生じ,日本語との差が16.0\%,英語との差が11.5\%であった.
  A問題は今回使用したデータセットの難易度の中で最も簡単な問題であるため,
  $Accuracy$の値に大きな差が生じにくいと予測していたが,予測とは異なる結果となった.
  その理由として,AtCoder\cite{AtCoder}において,日本語と英語の問題が準備されているため,
  簡単な問題であっても,これらの言語をコメントに含んだより多くの正解コードがアップロードされており,
  それらが学習に使用された可能性がある.

  その他の原因探索のため,実際に$Accuracy$の差が大きかった問題をいくつか確認した.
  図\ref{problem_212_A_en}の問題は,
  英語および日本語のデータセットでは,全ての推薦コードが全てのテストケースを通過しているが,
  中国語のデータセットでは,全ての推薦コードが全てのテストケースを通過していない例である.
  また,図\ref{recommend_212_A_en}は実際に問題212-Aにおける中国語のデータセットに対する推薦コードである.
  図\ref{recommend_212_A_en}より,単純な条件のみで構成されているものや,
  条件分岐の途中で推薦が打ち切られているもの,条件分岐の数が少ないものであった.
  
  この問題のように,条件が複数ある場合や,条件が複雑である場合,出力が文字列である場合に
  特に英語および日本語との$Accuracy$の差が大きくなる傾向があった.
  この原因として,AtCoderの文字列の出力形式がローマ字や英単語であるため,
  入力として使用した中国語のデータセットの文章中にローマ字や英単語が含まれており,
  それらがシンボルとして認識されなかった可能性や翻訳してデータセットを作成した影響が考えられる.
