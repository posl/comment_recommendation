\section{はじめに\label{intro}}
    近年IT需要の拡大に伴って,開発効率向上のために,タスク管理ツールやプロジェクト管理ツールをはじめ,
    様々な支援ツールを活用して開発が行われている.
    その中の一つに,2022年にGitHubが公開したGitHub Copilotがある(以下,Copilotと記述する).
    Copilotは大規模言語モデルをベースとしたコード推薦ツールの一種であり,仕様を記述したコメントや,
    記述中のプログラムをもとに開発者に対してコードやライブラリを推薦する.
    そのためフルスクラッチの必要がなく,開発コストの削減が見込まれる.

    一方,大規模な事前学習済み言語モデルは,入力によって出力が大きく異なることが知られている\cite{Yao2022ACL}.
    そのため,言語モデルの能力を最大限引き出すためには,適切なコメントを入力する必要がある.
    
    現在世界では7000語以上の言語が存在する\cite{Ethnologue}と言われているが,言語によって使用頻度は異なる.
    そのため,コメントに異なる言語を使用することで,学習データ数の不均衡などにより学習結果のバイアスにつながる可能性があると考えた.
    そこで本研究では,言語間の学習データ量の違いに着目し,言語の違いがCopilotの性能に与える影響に関して調査を行った.

    以降,第2章で関連研究および本研究の目的,
    第3章で実験設計について述べる.
    第4章で調査結果と追加実験の必要性を示し,第5章で考察を示す.
    最後に第6章で結論と今後の課題について述べる.
