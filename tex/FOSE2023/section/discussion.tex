\section{考察\label{consideration}}
  \ref{result}節で示した結果について,考察を行う.
  表\ref{Med_accuracy}より,中国語のデータセットは,英語のデータセットに比べて,日本語のデータセットとの$Accuracy$の差が大きかった.
  中でも,A問題に関しては,日本語との差が16.0\%,英語との差が11.5\%であった.
  A問題は今回使用したデータセットの難易度の中で最も簡単な問題であるため,
  $Accuracy$の値に大きな差が生じにくいと予測していたが,予測とは異なる結果となった.
  その理由として,AtCoder\cite{AtCoder}において,日本語と英語の問題が準備されているため,
  簡単な問題であっても,これらの言語をコメントに含んだより多くの正解プログラムがGitHub上にアプロードされており,
  それらが学習に使用された可能性がある.
  その他の原因探索のため,実際に$Accuracy$の差が大きかった問題をいくつか確認した.
  表\ref{problem_212_A_en}の問題は,
  英語および日本語のデータセットでは,全ての推薦スクリプトが全てのテストケースを通過しているが,
  中国語のデータセットでは,全ての推薦スクリプトが全てのテストケースを通過していない例である.
  実際に問題212-Aにおける中国語のデータセットに対する推薦スクリプトを見ると,
  単純な条件のみで構成されているものや,
  条件分岐の途中で推薦が打ち切られているもの,条件分岐の数が少ないものであった.
  この問題のように,条件が複数ある場合や,条件が複数ある場合,出力が文字列である場合に
  特に英語および日本語との$Accuracy$の差が大きくなる傾向があった.
  この原因として,AtCoderの文字列の出力形式がローマ字や英単語であるため,
  入力として使用した中国語のデータセットの文章中にローマ字や英単語が含まれており,
  それらがシンボルとして認識されなかった可能性がある.
