\section{結果\label{result}}
  \begin{table}[t]
    \centering
    \caption{英語における$Accuracy$}
    \label{En_accuracy}
    \scalebox{0.6}{
    \begin{tabular}{|c|c|c|c|c|c|c|c|c|}
      \hline
      {} & \multicolumn{2}{c|}{A} & \multicolumn{2}{c|}{B} & \multicolumn{2}{c|}{C} & \multicolumn{2}{c|}{D} \\
      \hline
      1 & 63.7 & 864/1357 & 51.0 & 860/1686 & 28.3 & 482/1702 & 10.4 & 172/1657 \\
      \hline
      2 & 65.4 & 878/1343 & 50.5 & 858/1698 & 26.7 & 471/1761 & 10.0 & 182/1813 \\
      \hline
      3 & 63.6 & 856/1346 & 50.9 & 811/1594 & 26.6 & 438/1648 & 9.31 & 151/1622 \\
      \hline
      4 & 60.4 & 819/1355 & 51.6 & 842/1631 & 31.3 & 499/1596 & 11.0 & 167/1524 \\
      \hline
      5 & 60.4 & 831/1375 & 50.5 & 857/1698 & 31.1 & 552/1774 & 12.0 & 217/1812 \\
      \hline
      \end{tabular}
    }
  \end{table}

  \begin{table}[t]
    \centering
    \caption{日本語における$Accuracy$}
    \label{Ja_accuracy}
    \scalebox{0.6}{
    \begin{tabular}{|c|c|c|c|c|c|c|c|c|}
      \hline
      {} & \multicolumn{2}{c|}{A} & \multicolumn{2}{c|}{B} & \multicolumn{2}{c|}{C} & \multicolumn{2}{c|}{D} \\
      \hline
      1 & 68.7 & 857/1248 & 52.8 & 872/1651 & 28.9 & 482/1668 & 10.0 & 170/1697 \\
      \hline
      2 & 67.4 & 840/1246 & 52.9 & 824/1559 & 28.4 & 454/1597 & 10.2 & 155/1526 \\
      \hline
      3 & 68.1 & 842/1237 & 54.3 & 879/1619 & 28.5 & 449/1577 & 11.0 & 153/1391 \\
      \hline
      4 & 68.4 & 860/1258 & 57.6 & 872/1515 & 33.1 & 439/1328 & 12.8 & 125/974 \\
      \hline
      5 & 68.0 & 849/1249 & 54.5 & 922/1693 & 32.7 & 592/1813 & 11.4 & 208/1832 \\
      \hline
      \end{tabular}
    }
  \end{table}
  
  \begin{table}[t]
    \centering
    \caption{中国語における$Accuracy$}
    \label{Zh_accuracy}
    \scalebox{0.6}{
    \begin{tabular}{|c|c|c|c|c|c|c|c|c|}
      \hline
      {} & \multicolumn{2}{c|}{A} & \multicolumn{2}{c|}{B} & \multicolumn{2}{c|}{C} & \multicolumn{2}{c|}{D} \\
      \hline
      1 & 52.8 & 758/1435 & 41.2 & 724/1756 & 23.2 & 415/1792 & 8.11 & 144/1775 \\
      \hline
      2 & 52.5 & 737/1403 & 44.0 & 749/1704 & 22.5 & 388/1724 & 8.00 & 138/1727 \\
      \hline
      3 & 52.1 & 747/1433 & 40.4 & 701/1734 & 23.4 & 420/1793 & 8.44 & 150/1777 \\
      \hline
      4 & 52.1 & 743/1426 & 42.3 & 736/1738 & 23.9 & 427/1789 & 8.99 & 158/1758 \\
      \hline
      5 & 52.2 & 756/1447 & 43.1 & 751/1744 & 22.6 & 406/1800 & 7.84 & 137/1748 \\
      \hline
      \end{tabular}
    }
  \end{table}

  \begin{table}[t]
    \centering
    \caption{各言語の$Accuracy$の中央値}
    \label{Med_accuracy}
    \scalebox{0.9}{
    \begin{tabular}{|c|c|c|c|c|}
      \hline
      {} & A & B & C & D \\
      \hline
      英語 & 63.6\% & 50.9\% & 28.3\% & 10.4\% \\
      \hline
      日本語 & 68.1\% & 54.3\% & 28.9\% & 11.0\% \\
      \hline
      中国語 & 52.2\% & 42.3\% & 23.2\% & 8.11\% \\
      \hline
      \end{tabular}
    }
  \end{table}

  本章では,RQと結果を示す.
  \begin{description}
    \item[\textbf{RQ}] \textbf{入力する言語の違いによって,Copilotの性能にどのような影響を与えるのか}
      \item[結果]
      
      表\ref{En_accuracy},表\ref{Ja_accuracy},および表\ref{Zh_accuracy}は
      それぞれ英語,日本語,および中国語における$Accuracy$を示したものである.
      これらの表から,生成毎に$Accuracy$にばらつきが生じているため,Copilotが毎回異なるスクリプトを推薦していることがわかる.
      また,推薦回数を重ねるほど$Accuracy$の値が単調に増加したり,減少したりすることはなかった.
      表\ref{En_accuracy}より,英語のデータセットにおいて,A問題では最大5.0\%,B問題では最大1.1\%,C問題では最大4.7\%,D問題では最大2.7\%の差が生じた.
      また,表\ref{Ja_accuracy}より,日本語のデータセットにおいて,A問題では最大1.3\%,B問題では最大4.8\%,C問題では最大4.7\%,D問題では最大2.8\%,
      表\ref{Zh_accuracy}より,中国語のデータセットにおいては,A問題で最大0.70\%,B問題で最大3.6\%,C問題で最大1.4\%,D問題で最大1.2\%の差が生じた.
      各言語の差が3.0\%以上だった問題の最大値に着目すると,推薦された全スクリプト数が5回のうち最も少なく,
      このことから,推薦スクリプトを最大10個取得することで,かえって$Accuracy$が低下する可能性があることが考えられる.
      また,難易度別に推薦された全スクリプト数に着目すると,A問題はB, C, D問題に比べて,推薦された全スクリプト数が最も少なかった.
      これは問題が単純であるが故に,Copilotがスクリプトを推薦する際に,生成されるスクリプトの多様性が低くなるためであると考えられる.
      さらに,言語別で推薦された全スクリプト数を平均すると,英語のデータセットでは,A問題が1355問,B問題が1661問,C問題が1696問,D問題が1686問,
      日本語のデータセットでは,A問題が1247問,B問題が1607問,C問題が1596問,D問題が1484問,
      中国語のデータセットでは,A問題が1428問,B問題が1735問,C問題が1779問,D問題が1757問であった.
      これらの結果から,日本語のデータセットでは,英語のデータセットと中国語のデータセットに比べて,推薦された全スクリプト数が少なかった.
      加えて,全難易度で日本語のデータセットの$Accuracy$が最も高かった.
      これは,日本語での入力により,Copilotがより最適なスクリプトのみを推薦していることを示している.
      この理由として,AtCoder\cite{AtCoder}が日本で運営されているため,
      より多くの日本人がAtCoderを使用しており,日本語のコメントを含んだ回答が学習時に多く使用されたためであると考えられる.      
      また,表\ref{Med_accuracy}は,各言語の難易度別の$Accuracy$の中央値を示したものである.   
      表\ref{Med_accuracy}より,中国語のデータセットは,英語のデータセットに比べて,日本語のデータセットとの$Accuracy$の差が大きかった.
      これは,AtCoder\cite{AtCoder}において,日本語と英語の問題が準備されているため,
      日本語や英語をコメントの含んだより多くの正解プログラムが学習時に使用されたためであると考えられる.

      %各言語の難易度別の$Accuracy$の中央値を比較すると,英語のデータセットでは,A問題が63.6\%,B問題が50.9\%,C問題が28.3\%,D問題が10.4\%,
      %日本語のデータセットでは,A問題が68.1\%,B問題が54.3\%,C問題が28.9\%,D問題が11.0\%,
      %中国語のデータセットでは,A問題が52.2\%,B問題が42.3\%,C問題が23.2\%,D問題が8.11\%であった.
      %これらの結果から,中国語のデータセットは,英語のデータセットに比べて,日本語のデータセットとの$Accuracy$の差が大きかった.
      %これは,AtCoder\cite{AtCoder}において,日本語と英語の問題が準備されているため,
      %日本語や英語をコメントの含んだより多くの正解プログラムが学習時に使用されたためであると考えられる.
      
      
  \end{description}