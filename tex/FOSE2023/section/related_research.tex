\section{背景と目的\label{related_research}}

\subsection{本研究の目的}
  急速に進むIT化に伴い,事前学習済み大規模言語モデルの活用が進んでおり,
  今後ますます大規模言語モデルが組み込まれたシステムが拡大していくことが予想される.
  その中の一つであるCopilotに対しても研究が盛んに行われている\cite{Yao2022ACL}\cite{Nguyen2022MSR}\cite{Sobania2022GECCO}\cite{Dakhel2022arXiv}\cite{Vaithilingam2022CHI}.
  一方で,大規模な事前学習済み言語モデルは,
  入力によって出力が大きく異なることが知られている\cite{Yao2022ACL}.
  そのため,言語モデルの能力を最大限引き出すためには,適切なコメントを入力する必要がある.
  コメントは主要な要素として入力言語と入力内容で構成されるが,
  入力言語の違いによる性能への影響については調査が行われていない.
  そこで,入力言語の違いによる性能への影響を調査することで,
  システムの性能を最大限引き出すための手がかりを得ることができると考えた.
  本稿では,日本語,英語,および中国語の3言語を入力とした場合のそれぞれのCopilotの性能を比較する.
  調査課題を以下に示す.

  
  \begin{itemize}
    \item[\textbf{RQ}] \textbf{入力する言語の違いによって,Copilotの性能にどのような影響を与えるのか}
      \item[目的]{%近年,大規模な事前学習済み言語モデルを用いたシステムが拡大しており,
      %今後も拡大していくことが予想される.
      %しかし,
      現状,大規模な事前学習済み言語モデルは,入力によって出力が大きく異なり,
      主な要素としては入力内容および入力言語がある.
      本研究では,後者の言語に着目し,入力言語によってCopilotの性能に差が生じるのか明らかにすることで,
      今後のCopilotの最適な活用についての知見を得る.}
  \end{itemize}

\subsection{プロンプトエンジニアリング}
  コード生成におけるプロントエンジニアリングとは,モデルに対してどのようなプロンプトを入力として与えれば生成精度がより向上するかを探索する手法である.
  プロンプトとはモデルに対して与える入力のことで,モデルに対して与える入力としてはコードの仕様やプログラムそのものなどがある.
  モデルは与えられたプロンプトをもとに,次にどのようなをコードを生成するかを予測する.

\subsection{関連研究}
  Yaoら\cite{Yao2022ACL}は,プロンプトとして入力するサンプルの入力順序の違いによって,GPT-3のような
  大規模な事前学習済み言語モデルの性能にどのような影響を与えるのか調査を行った.
  その結果,サンプルの入力順序によって性能にばらつきが生じ,
  これがモデルサイズに関係なく発生すること,サンプルの特定のセットに関係なく発生すること,
  およびあるモデルには優れた学習順序であっても別のモデルには適用できないことを示した.
  また,各モデルに対して優れた性能を示すプロンプトの探索手法としてエントロピーベースでの探索手法を提案した.
  その結果,エントロピーベースでの探索手法は,ランダムにプロンプトを選択するよりも優れた性能を示した.

  Nguyen\cite{Nguyen2022MSR}らは,Copilotのコード推薦の精度および推薦されたコードの品質について,
  異なるプログラミング言語で調査を行った.
  LeetCodeの問題に対してEasy, Medium, Hardの3つの難易度で調査を行ったところ,
  Easyでは全ての言語が全テストケースを通過し,全体として,Java, Python, C, Javascriptの順に高い精度を示した.
  また,推薦されたコードの品質については,言語間における差はほとんど生じず,
  判読性は高いことを示した.


  